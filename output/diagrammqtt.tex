\documentclass[a4paper,10pt]{article}
\usepackage[utf8]{inputenc}
\usepackage[margin=1cm]{geometry}
\usepackage{tikz}
\usetikzlibrary{shapes, arrows, positioning, calc, backgrounds}
\usepackage{float}

\title{Diagram Komunikasi MQTT Sistem IoT}
\author{ESP8266 \& Python Application}
\date{\today}

\begin{document}
\maketitle

\section{Diagram Komunikasi MQTT}

\begin{figure}[H]
\centering
\begin{tikzpicture} [
    node distance=4cm,
    python/.style={rectangle, draw=black, thick, fill=blue!20, minimum width=3cm, minimum height=2.5cm, font=\footnotesize, text centered, rounded corners=3pt},
    mqtt/.style={rectangle, draw=black, thick, fill=orange!20, minimum width=3cm, minimum height=2.5cm, font=\footnotesize, text centered, rounded corners=3pt},
    esp/.style={rectangle, draw=black, thick, fill=green!20, minimum width=3cm, minimum height=2.5cm, font=\footnotesize, text centered, rounded corners=3pt},
    arrow/.style={->, thick, >=stealth}
]

% Layer 1: Draw background connections first
\draw[arrow, blue!60, line width=2pt] (0,0) -- (4,0);
\draw[arrow, red!60, line width=2pt] (4,0) -- (8,0);  
\draw[arrow, green!60, line width=2pt] (8,0) -- (4,0);
\draw[arrow, purple!60, line width=2pt] (4,0) -- (0,0);

% Layer 2: Draw nodes on top
\node[python] (python_app) at (0,0) {
    \textbf{Python App}\\[0.3cm]
    🖥️ GUI \& Analysis\\[0.2cm]
    $\bullet$ Real-time Monitoring\\
    $\bullet$ Data Processing\\
    $\bullet$ Control Interface
};

\node[mqtt] (mqtt_broker) at (4,0) {
    \textbf{MQTT Broker}\\[0.3cm]
    ☁️ HiveMQ Cloud\\[0.2cm]
    $\bullet$ Message Routing\\
    $\bullet$ Real-time Delivery\\
    $\bullet$ Topic Management
};

\node[esp] (esp_device) at (8,0) {
    \textbf{ESP8266}\\[0.3cm]
    📡 HC-SR04 Sensor\\[0.2cm]
    $\bullet$ Distance Measurement\\
    $\bullet$ WiFi Connection\\
    $\bullet$ JSON Data Format
};

% Layer 3: Add labels on top of arrows
\node[font=\small, text=blue, fill=white, inner sep=2pt, draw=blue, thick, rounded corners=2pt] at (2,0.5) {Commands};
\node[font=\small, text=red, fill=white, inner sep=2pt, draw=red, thick, rounded corners=2pt] at (6,0.5) {Forward};
\node[font=\small, text=green, fill=white, inner sep=2pt, draw=green, thick, rounded corners=2pt] at (6,-0.5) {Sensor Data};
\node[font=\small, text=purple, fill=white, inner sep=2pt, draw=purple, thick, rounded corners=2pt] at (2,-0.5) {Data};

% Layer 4: Add topic labels
\node[font=\tiny, text=blue, fill=yellow!30, inner sep=1pt, draw=blue] at (2,1.2) {sensor/distance/cmd};
\node[font=\tiny, text=green, fill=yellow!30, inner sep=1pt, draw=green] at (6,-1.2) {sensor/distance};

\end{tikzpicture}
\caption{Arsitektur Komunikasi MQTT Sederhana}
\end{figure}

\subsection{Penjelasan Alur Komunikasi}

\begin{enumerate}
\item \textbf{📤 Commands}: Python App mengirim perintah START/STOP ke MQTT Broker
\item \textbf{🔄 Forward}: MQTT Broker meneruskan perintah ke ESP8266
\item \textbf{📡 Sensor Data}: ESP8266 mengirim data jarak JSON ke MQTT Broker
\item \textbf{📥 Data}: MQTT Broker mengirim data ke Python App untuk analisis
\end{enumerate}

\subsection{MQTT Topics}
\begin{itemize}
\item \texttt{sensor/distance/cmd} - Topik untuk perintah kontrol
\item \texttt{sensor/distance} - Topik untuk data sensor
\end{itemize}

\subsection{Keunggulan Komunikasi MQTT}
\begin{itemize}
\item \textbf{Real-time}: Latensi rendah untuk monitoring langsung
\item \textbf{Scalable}: Mudah menambah perangkat IoT
\item \textbf{Reliable}: Quality of Service (QoS) terjamin
\item \textbf{Internet-based}: Akses dari jarak jauh
\end{itemize}

\end{document}