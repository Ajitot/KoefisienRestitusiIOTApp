\documentclass[12pt,a4paper]{article}
\usepackage[utf8]{inputenc}
\usepackage[bahasa]{babel}
\usepackage{geometry}
\usepackage{graphicx}
\usepackage{amsmath}
\usepackage{amsfonts}
\usepackage{amssymb}
\usepackage{fancyhdr}
\usepackage{listings}
\usepackage{xcolor}
\usepackage{booktabs}
\usepackage{float}
\usepackage{hyperref}
\usepackage{titlesec}
\usepackage{enumitem}
\usepackage{multicol}
\usepackage{array}
\usepackage{longtable}

% Page setup
\geometry{margin=2.5cm}
\pagestyle{fancy}
\fancyhf{}
\fancyhead[L]{Modul Praktikum Koefisien Restitusi IoT}
\fancyhead[R]{Teknik Elektro}
\fancyfoot[C]{\thepage}

% Colors for code highlighting
\definecolor{codegreen}{rgb}{0,0.6,0}
\definecolor{codegray}{rgb}{0.5,0.5,0.5}
\definecolor{codepurple}{rgb}{0.58,0,0.82}
\definecolor{backcolour}{rgb}{0.95,0.95,0.92}

% Code listing style
\lstdefinestyle{mystyle}{
    backgroundcolor=\color{backcolour},   
    commentstyle=\color{codegreen},
    keywordstyle=\color{magenta},
    numberstyle=\tiny\color{codegray},
    stringstyle=\color{codepurple},
    basicstyle=\ttfamily\footnotesize,
    breakatwhitespace=false,         
    breaklines=true,                 
    captionpos=b,                    
    keepspaces=true,                 
    numbers=left,                    
    numbersep=5pt,                  
    showspaces=false,                
    showstringspaces=false,
    showtabs=false,                  
    tabsize=2
}
\lstset{style=mystyle}

% Title formatting
\titleformat{\section}{\large\bfseries}{\thesection}{1em}{}
\titleformat{\subsection}{\normalsize\bfseries}{\thesubsection}{1em}{}

\begin{document}

% Cover Page
\begin{titlepage}
    \centering
    \vspace*{2cm}
    
    {\huge\bfseries MODUL PRAKTIKUM\\[0.5cm]}
    {\Large ANALISIS KOEFISIEN RESTITUSI\\MENGGUNAKAN SISTEM IoT\\[0.3cm]}
    {\large BERBASIS ESP8266/ESP32 DAN SENSOR HC-SR04\\[2cm]}
    
    % \includegraphics[width=0.3\textwidth]{logo_univ.png}\\[1cm]
    
    {\large\bfseries Program Studi Teknik Elektro\\[0.5cm]}
    {\large Mata Kuliah: Wireless Power Transfer / IoT dan Sistem Tertanam\\[0.3cm]}
    {\large Semester: VI/VII\\[0.3cm]}
    {\large Waktu: 3 × 50 menit\\[2cm]}
    
    {\large\bfseries Disusun oleh:\\[0.5cm]}
    {\large Aji Muhamad Pranata\\[0.2cm]}
    {\large NIM: 1217030004\\[3cm]}
    
    {\large Tahun Akademik 2024/2025}
    
\end{titlepage}

\newpage
\tableofcontents
\newpage

\section{TUJUAN PRAKTIKUM}

\subsection{Tujuan Umum}
Mahasiswa mampu memahami dan mengimplementasikan sistem IoT untuk pengukuran fisika secara real-time dengan menganalisis koefisien restitusi bola menggunakan sensor ultrasonik.

\subsection{Tujuan Khusus}
\begin{enumerate}
    \item Memahami prinsip kerja sensor ultrasonik HC-SR04
    \item Mengimplementasikan komunikasi MQTT pada ESP8266/ESP32
    \item Membangun sistem monitoring real-time menggunakan Python dan Tkinter
    \item Menganalisis data sensor untuk menghitung koefisien restitusi
    \item Memahami konsep fisika tentang tumbukan elastis dan inelastis
    \item Menggunakan signal processing untuk deteksi puncak pantulan
\end{enumerate}

\section{DASAR TEORI}

\subsection{Koefisien Restitusi}
Koefisien restitusi (e) adalah ukuran elastisitas tumbukan antara dua benda. Untuk bola yang dijatuhkan dari ketinggian tertentu:

\begin{equation}
e = \sqrt{\frac{h_2}{h_1}}
\end{equation}

dimana:
\begin{itemize}
    \item $h_1$ = tinggi pantulan pertama (cm)
    \item $h_2$ = tinggi pantulan kedua (cm)  
    \item $e$ = koefisien restitusi ($0 \leq e \leq 1$)
\end{itemize}

\textbf{Klasifikasi tumbukan:}
\begin{itemize}
    \item $e = 1$: Tumbukan elastis sempurna (tidak ada kehilangan energi)
    \item $e = 0$: Tumbukan inelastis sempurna (kehilangan energi maksimum)
    \item $0 < e < 1$: Tumbukan semi-elastis (kondisi nyata)
\end{itemize}

\subsection{Sensor HC-SR04}
Sensor ultrasonik yang mengukur jarak dengan prinsip:
\begin{enumerate}
    \item Mengirim gelombang ultrasonik (40 kHz)
    \item Mengukur waktu pantul (echo)
    \item Menghitung jarak: $d = \frac{t \times v}{2}$
\end{enumerate}

\textbf{Spesifikasi teknis:}
\begin{itemize}
    \item Jangkauan: 2cm - 400cm
    \item Akurasi: ±3mm
    \item Sudut pengukuran: <15°
    \item Tegangan operasi: 5V DC
    \item Arus: 15mA
\end{itemize}

\subsection{Protokol MQTT}
Message Queuing Telemetry Transport untuk komunikasi IoT:
\begin{itemize}
    \item \textbf{Publisher}: ESP8266/ESP32 (sensor data)
    \item \textbf{Subscriber}: Aplikasi Python (data processing)
    \item \textbf{Broker}: HiveMQ (cloud broker)
    \item \textbf{Topic}: \texttt{sensor/distance}
\end{itemize}

\subsection{Signal Processing}
\begin{itemize}
    \item \textbf{Low-pass filter}: Mengurangi noise dari sensor menggunakan filter Butterworth
    \item \textbf{Peak detection}: Mendeteksi puncak pantulan bola menggunakan algoritma \texttt{find\_peaks}
    \item \textbf{Threshold}: Ambang batas minimum untuk validasi pantulan
\end{itemize}

\section{ALAT DAN BAHAN}

\subsection{Hardware}
\begin{enumerate}
    \item ESP8266/ESP32 development board (1 unit)
    \item Sensor HC-SR04 (1 unit)
    \item Kabel jumper male-female (4 buah)
    \item Breadboard (1 unit)
    \item Kabel USB (1 unit)
    \item Bola untuk pengujian:
    \begin{itemize}
        \item Bola bekel
        \item Bola tenis meja  
        \item Bola tenis lapangan
        \item Bola plastik
        \item Bola sepak karet
    \end{itemize}
\end{enumerate}

\subsection{Software}
\begin{enumerate}
    \item Arduino IDE dengan library:
    \begin{itemize}
        \item WiFi (ESP8266WiFi/WiFi)
        \item PubSubClient
        \item ArduinoJson
    \end{itemize}
    \item Python 3.x dengan library:
    \begin{itemize}
        \item tkinter (GUI)
        \item matplotlib (plotting)
        \item pandas (data processing)
        \item scipy (signal processing)
        \item paho-mqtt (MQTT client)
        \item numpy (numerical computing)
    \end{itemize}
    \item MQTT Broker (HiveMQ - cloud)
\end{enumerate}

\subsection{Tools Pendukung}
\begin{enumerate}
    \item Penggaris/meteran (untuk kalibrasi)
    \item Tripod atau penyangga untuk sensor
    \item Komputer/laptop dengan koneksi internet
\end{enumerate}

\section{PROSEDUR PRAKTIKUM}

\subsection{Langkah 1: Setup Hardware}

\subsubsection{Koneksi ESP8266 dengan HC-SR04}
\begin{table}[H]
\centering
\begin{tabular}{@{}cc@{}}
\toprule
\textbf{HC-SR04} & \textbf{ESP8266} \\
\midrule
VCC & 3.3V/5V \\
GND & GND \\
TRIG & D1 (GPIO 5) \\
ECHO & D2 (GPIO 4) \\
\bottomrule
\end{tabular}
\caption{Koneksi ESP8266 dengan HC-SR04}
\end{table}

\subsubsection{Koneksi ESP32 dengan HC-SR04}
\begin{table}[H]
\centering
\begin{tabular}{@{}cc@{}}
\toprule
\textbf{HC-SR04} & \textbf{ESP32} \\
\midrule
VCC & 5V \\
GND & GND \\
TRIG & GPIO 14 \\
ECHO & GPIO 27 \\
\bottomrule
\end{tabular}
\caption{Koneksi ESP32 dengan HC-SR04}
\end{table}

\subsubsection{Pemasangan Sensor}
\begin{enumerate}
    \item Pasang sensor pada ketinggian 35cm dari lantai
    \item Arahkan sensor vertikal ke bawah
    \item Pastikan area bebas dari halangan dalam radius 50cm
    \item Gunakan tripod atau penyangga yang stabil
\end{enumerate}

\subsection{Langkah 2: Programming ESP8266/ESP32}

\subsubsection{Konfigurasi Arduino IDE}
\begin{enumerate}
    \item Buka Arduino IDE
    \item Install ESP8266/ESP32 board package
    \item Install library yang diperlukan:
    \begin{itemize}
        \item PubSubClient by Nick O'Leary
        \item ArduinoJson by Benoit Blanchon
    \end{itemize}
\end{enumerate}

\subsubsection{Upload Program}
\begin{enumerate}
    \item Copy code dari file \texttt{src/main.cpp}
    \item Konfigurasi WiFi:
\end{enumerate}

\begin{lstlisting}[language=C++, caption=Konfigurasi WiFi]
const char* ssid = "NAMA_WIFI_ANDA";
const char* password = "PASSWORD_WIFI_ANDA";
\end{lstlisting}

\begin{enumerate}[resume]
    \item Pilih board yang sesuai (ESP8266/ESP32)
    \item Pilih port COM yang benar
    \item Upload program ke ESP
    \item Monitor Serial untuk debugging
\end{enumerate}

\subsection{Langkah 3: Setup Software Python}

\subsubsection{Install Dependencies}
\begin{lstlisting}[language=bash, caption=Install Python Libraries]
pip install tkinter matplotlib pandas scipy paho-mqtt numpy openpyxl
\end{lstlisting}

\subsubsection{Jalankan Aplikasi}
\begin{lstlisting}[language=bash, caption=Menjalankan Aplikasi Python]
cd python
python main.py
\end{lstlisting}

\subsection{Langkah 4: Kalibrasi Sistem}

\subsubsection{Pengaturan Tinggi Sensor}
\begin{enumerate}
    \item Klik tombol "Tinggi Sensor: 35cm"
    \item Masukkan tinggi aktual sensor dari lantai
    \item Validasi dengan mengukur jarak objek yang diketahui
\end{enumerate}

\subsubsection{Test Koneksi MQTT}
\begin{enumerate}
    \item Pastikan status menunjukkan "Terhubung ke MQTT"
    \item Cek data masuk di tabel real-time
    \item Verifikasi komunikasi dua arah dengan ESP
\end{enumerate}

\subsubsection{Validasi Pembacaan}
\begin{enumerate}
    \item Letakkan objek di berbagai jarak
    \item Cek keakuratan pembacaan sensor
    \item Bandingkan dengan pengukuran manual
\end{enumerate}

\subsection{Langkah 5: Pengumpulan Data}

\subsubsection{Persiapan Pengukuran}
\begin{enumerate}
    \item Pilih jenis bola menggunakan dropdown
    \item Set parameter deteksi sesuai karakteristik bola
    \item Reset data pengukuran sebelumnya
    \item Pastikan area pengukuran bebas gangguan
\end{enumerate}

\subsubsection{Eksekusi Pengukuran}
\begin{enumerate}
    \item Klik "Mulai Pengumpulan Data"
    \item Jatuhkan bola dari ketinggian ≈30cm
    \item Biarkan bola memantul sampai berhenti
    \item Klik "Hentikan Pengumpulan"
    \item Catat durasi dan jumlah pantulan
\end{enumerate}

\subsubsection{Pengulangan untuk Bola Lain}
\begin{enumerate}
    \item Reset data sebelumnya
    \item Ganti jenis bola di pengaturan
    \item Ulangi prosedur pengukuran
    \item Dokumentasikan setiap percobaan
\end{enumerate}

\subsection{Langkah 6: Analisis Data}

\subsubsection{Perhitungan Koefisien}
\begin{enumerate}
    \item Klik "Hitung Koefisien Restitusi"
    \item Tunggu proses analisis selesai
    \item Review hasil di panel analisis
    \item Catat nilai koefisien dan statistik
\end{enumerate}

\subsubsection{Interpretasi Hasil}
Analisis meliputi:
\begin{itemize}
    \item Koefisien restitusi rata-rata
    \item Standar deviasi
    \item Klasifikasi material
    \item Persentase retensi energi
    \item Tren penurunan tinggi pantulan
\end{itemize}

\subsubsection{Export dan Dokumentasi}
\begin{enumerate}
    \item Simpan data mentah ke format Excel
    \item Export grafik ke format PNG/PDF
    \item Simpan hasil analisis lengkap ke file teks
    \item Buat screenshot interface untuk laporan
\end{enumerate}

\section{PENGAMATAN DAN ANALISIS}

\subsection{Tabel Pengamatan}
\begin{longtable}{|p{2cm}|p{2cm}|p{2cm}|p{2cm}|p{3cm}|}
\hline
\textbf{Jenis Bola} & \textbf{Pantulan ke-} & \textbf{Tinggi (cm)} & \textbf{Koefisien (e)} & \textbf{Retensi Energi (\%)} \\
\hline
\multirow{3}{*}{Bekel} & 1 & & & \\
\cline{2-5}
& 2 & & & \\
\cline{2-5}
& 3 & & & \\
\hline
\multirow{3}{*}{Tenis Meja} & 1 & & & \\
\cline{2-5}
& 2 & & & \\
\cline{2-5}
& 3 & & & \\
\hline
\multirow{3}{*}{Plastik} & 1 & & & \\
\cline{2-5}
& 2 & & & \\
\cline{2-5}
& 3 & & & \\
\hline
\end{longtable}

\subsection{Pertanyaan Analisis}

\subsubsection{Analisis Koefisien}
\begin{enumerate}
    \item Bandingkan koefisien restitusi antar jenis bola
    \item Bola mana yang paling elastis? Jelaskan alasannya
    \item Hubungkan hasil dengan material penyusun bola
    \item Analisis konsistensi koefisien antar pantulan
\end{enumerate}

\subsubsection{Analisis Energi}
\begin{enumerate}
    \item Hitung persentase energi yang hilang setiap pantulan
    \item Identifikasi faktor penyebab kehilangan energi
    \item Analisis trend energi pada pantulan berturut-turut
    \item Bandingkan efisiensi energi antar material
\end{enumerate}

\subsubsection{Analisis Signal Processing}
\begin{enumerate}
    \item Mengapa diperlukan filter low-pass pada data sensor?
    \item Bagaimana pengaruh threshold terhadap deteksi pantulan?
    \item Apa keuntungan menggunakan algoritma peak detection?
    \item Analisis noise dan cara mengatasinya
\end{enumerate}

\subsection{Contoh Perhitungan}

Untuk bola dengan data pantulan:
\begin{itemize}
    \item $h_1 = 25.3$ cm
    \item $h_2 = 18.7$ cm
\end{itemize}

\textbf{Perhitungan koefisien restitusi:}
\begin{align}
e &= \sqrt{\frac{h_2}{h_1}} \\
&= \sqrt{\frac{18.7}{25.3}} \\
&= \sqrt{0.739} \\
&= 0.860
\end{align}

\textbf{Analisis energi:}
\begin{align}
\text{Retensi energi} &= e^2 \times 100\% \\
&= 0.860^2 \times 100\% \\
&= 74.0\%
\end{align}

\begin{align}
\text{Kehilangan energi} &= 100\% - 74.0\% \\
&= 26.0\%
\end{align}

\section{TROUBLESHOOTING}

\subsection{Masalah Koneksi ESP}
\begin{table}[H]
\centering
\begin{tabular}{|p{5cm}|p{8cm}|}
\hline
\textbf{Masalah} & \textbf{Solusi} \\
\hline
ESP tidak terhubung WiFi & Cek SSID dan password, pastikan sinyal kuat, reset ESP \\
\hline
Data tidak masuk ke Python & Cek koneksi MQTT broker, pastikan topic sama, periksa firewall \\
\hline
Pembacaan sensor tidak akurat & Cek koneksi kabel, pastikan sensor tidak tertutup, kalibrasi ulang \\
\hline
Pantulan tidak terdeteksi & Turunkan threshold deteksi, pastikan bola memantul cukup tinggi \\
\hline
\end{tabular}
\caption{Troubleshooting Umum}
\end{table}

\subsection{Masalah Software Python}
\begin{table}[H]
\centering
\begin{tabular}{|p{5cm}|p{8cm}|}
\hline
\textbf{Masalah} & \textbf{Solusi} \\
\hline
Program Python error & Install ulang dependencies, cek versi Python (minimal 3.6) \\
\hline
GUI tidak responsif & Restart aplikasi, cek penggunaan memori sistem \\
\hline
Export file gagal & Pastikan direktori memiliki permission write, tutup file yang sedang dibuka \\
\hline
Grafik tidak update & Klik refresh manual, restart koneksi MQTT \\
\hline
\end{tabular}
\caption{Troubleshooting Software}
\end{table}

\section{TUGAS DAN EVALUASI}

\subsection{Tugas Praktikum}

\subsubsection{Pengukuran Dasar (30\%)}
\begin{enumerate}
    \item Ukur koefisien restitusi 5 jenis bola berbeda
    \item Buat tabel perbandingan hasil pengukuran
    \item Analisis perbedaan karakteristik antar bola
    \item Dokumentasikan setiap tahap pengukuran
\end{enumerate}

\subsubsection{Analisis Lanjutan (40\%)}
\begin{enumerate}
    \item Studi pengaruh ketinggian jatuh terhadap koefisien
    \item Analisis pengaruh parameter filter terhadap hasil
    \item Validasi hasil dengan perhitungan manual
    \item Buat grafik perbandingan koefisien vs material
\end{enumerate}

\subsubsection{Pengembangan Sistem (30\%)}
\begin{enumerate}
    \item Modifikasi parameter deteksi untuk optimasi
    \item Implementasi filter tambahan (optional)
    \item Proposal peningkatan akurasi sistem
    \item Evaluasi kelebihan dan kekurangan sistem
\end{enumerate}

\subsection{Format Laporan}

\subsubsection{Struktur Laporan}
\begin{enumerate}
    \item \textbf{Cover dan Identitas}
    \item \textbf{Tujuan dan Dasar Teori}
    \item \textbf{Prosedur yang Dilakukan}
    \item \textbf{Data dan Hasil Pengamatan}
    \item \textbf{Analisis dan Pembahasan}
    \item \textbf{Kesimpulan dan Saran}
    \item \textbf{Lampiran} (code, grafik, data Excel)
\end{enumerate}

\subsubsection{Kriteria Penilaian}
\begin{itemize}
    \item Kelengkapan data: 25\%
    \item Analisis dan pembahasan: 35\%
    \item Kesimpulan: 20\%
    \item Format dan kerapian: 20\%
\end{itemize}

\section{KESIMPULAN}

Praktikum ini memberikan pengalaman komprehensif dalam:

\subsection{Aspek IoT}
\begin{itemize}
    \item Implementasi komunikasi wireless menggunakan MQTT
    \item Pengembangan sistem monitoring real-time
    \item Integrasi sensor dengan cloud-based data processing
    \item Pemahaman arsitektur sistem IoT terdistribusi
\end{itemize}

\subsection{Aspek Fisika}
\begin{itemize}
    \item Pemahaman mendalam tentang koefisien restitusi
    \item Analisis energi dalam tumbukan elastis dan inelastis
    \item Karakterisasi material berdasarkan sifat elastisitas
    \item Aplikasi prinsip fisika dalam teknologi modern
\end{itemize}

\subsection{Aspek Programming}
\begin{itemize}
    \item Embedded programming dengan Arduino/ESP
    \item Desktop application development dengan Python
    \item Signal processing dan analisis data
    \item Integration testing dan debugging sistem
\end{itemize}

\section{PENGEMBANGAN LANJUTAN}

\subsection{Machine Learning Integration}
\begin{itemize}
    \item Prediksi koefisien berdasarkan material
    \item Klasifikasi otomatis jenis bola
    \item Optimasi parameter deteksi menggunakan AI
\end{itemize}

\subsection{IoT Dashboard}
\begin{itemize}
    \item Web-based monitoring dashboard
    \item Database storage untuk historical data
    \item Multi-sensor integration dan comparison
\end{itemize}

\subsection{Mobile Application}
\begin{itemize}
    \item Android/iOS app development
    \item Bluetooth/WiFi Direct communication
    \item Portable measurement system
\end{itemize}

\section{REFERENSI}

\begin{enumerate}
    \item Halliday, D., Resnick, R., \& Walker, J. (2017). \textit{Fundamentals of Physics}. 11th Edition. Wiley.
    
    \item Schwartz, M. (2016). \textit{Internet of Things with ESP8266}. Packt Publishing.
    
    \item McKinney, W. (2017). \textit{Python for Data Analysis}. 2nd Edition. O'Reilly Media.
    
    \item IEEE Standard for Internet of Things (IoT) Terminology. (2018). IEEE Std 2413-2019.
    
    \item MQTT Version 3.1.1 OASIS Standard. (2014). OASIS.
    
    \item Ultrasonic Sensor HC-SR04 Datasheet. (2020). ETC Electronic.
    
    \item Arduino Official Documentation. \url{https://docs.arduino.cc/}
    
    \item Python Official Documentation. \url{https://docs.python.org/3/}
\end{enumerate}

\section{LAMPIRAN}

\subsection{Lampiran A: Pinout Diagram ESP8266}
\begin{figure}[H]
\centering
\begin{verbatim}
ESP8266 NodeMCU Pinout:
┌─────────────────┐
│ A0        D0    │
│ RSV       D1 ──┼── TRIG (HC-SR04)
│ RSV       D2 ──┼── ECHO (HC-SR04)  
│ SD3       D3    │
│ SD2       D4    │
│ SD1       3V3 ──┼── VCC (HC-SR04)
│ CMD       GND ──┼── GND (HC-SR04)
│ SD0       D5    │
│ CLK       D6    │
│ GND       D7    │
│ 3V3       D8    │
│ EN        RX    │
│ RST       TX    │
│ GND       GND   │
│ VIN       3V3   │
└─────────────────┘
\end{verbatim}
\caption{Diagram Pinout ESP8266 NodeMCU}
\end{figure}

\subsection{Lampiran B: Format Pesan MQTT}
\begin{lstlisting}[language=json, caption=Format JSON MQTT Message]
{
  "timestamp": 1.23,
  "distance": 25.4,
  "device": "ESP8266_HCSR04",
  "uptime": 12345,
  "reading_time": 1.23
}
\end{lstlisting}

\subsection{Lampiran C: Rumus Fisika}
\begin{align}
\text{Koefisien Restitusi:} \quad e &= \sqrt{\frac{h_2}{h_1}} \\
\text{Energi Kinetik:} \quad E_k &= \frac{1}{2}mv^2 \\
\text{Energi Potensial:} \quad E_p &= mgh \\
\text{Kecepatan jatuh:} \quad v &= \sqrt{2gh} \\
\text{Waktu jatuh:} \quad t &= \sqrt{\frac{2h}{g}}
\end{align}

\subsection{Lampiran D: Kode Program ESP8266}
\lstinputlisting[language=C++, caption=Kode Program ESP8266 (main.cpp)]{src/main.cpp}

\subsection{Lampiran E: Kode Program Python}
\lstinputlisting[language=Python, caption=Kode Program Python (main.py), firstline=1, lastline=50]{python/main.py}

\vfill
\centering
\textbf{© 2024 - Modul Praktikum Koefisien Restitusi IoT}\\
\textbf{Program Studi Teknik Elektro}

\end{document}
